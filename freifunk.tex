\section{Freifunk}
\begin{frame}{}
    \begin{center}
        Freifunk
     \end{center}
\end{frame}

\begin{frame}{Freifunk}
    \begin{itemize}
        \item Offenes WLAN
        \begin{itemize}
            \item Wireless-Adhoc-Mesh Netzwerk
        \end{itemize}
        \item Lokaler Ableger der Freifunk-Bewegung (freifunk.net)
        \item Nicht-kommerzielle Initiative für freie Funknetzwerke
        \begin{itemize}
            \item[$\rightarrow$] Bürger investieren in Eigenregie Zeit, Geld und Enthusiasmus
        \end{itemize}
        \item Mehr als \glqq{}kostenloses Internet\grqq:
        \begin{itemize}
            \item Unabhängiges / dezentrales Netz
            \item Internet Technologie beim Bürger
            \item[$\rightarrow$] Freies Netzwerken
        \end{itemize}
    \item \texttt{\small{}(Irgendwann mal Teil vom Internet?)}
    \end{itemize}
\end{frame}

\begin{frame}{Das braucht man}
    \begin{itemize}
        \item Das Netz benutzen
        \begin{itemize}
            \item Handelübliches Wlan Gerät
        \end{itemize}
        \item Das Netz mit aufbauen
        \begin{itemize}
            \item Einen günstigen, unterstützten Router (ab ca. 17€)
            \item Eine spezielle Firmware
            \item Die Zustimmung zum \glqq{}Pico-Peering Agreement\grqq
                \footnote{Regelwerk, über grundsätzliche Eigenschaften des Freifunks}:
                \begin{enumerate}
                    \item Freier Transit
                    \item Offene Kommunikation
                    \item Keine Garantie (Haftungsausschluss)
                    \item Nutzungsbestimmungen
                    \item Lokale (individuelle) Zusätze
                \end{enumerate}
        \end{itemize}
    \end{itemize}
\end{frame}

