\section{Freifunk}

\begin{frame}{Freifunk}
    \begin{itemize}
        \item Lokaler Ableger der Freifunk-Bewegung (freifunk.net)
        \item Nicht-kommerzielle Initiative für freie Funknetzwerke
        \begin{itemize}
            \item[$\rightarrow$] Bürger investieren in Eigenregie Zeit, Geld und Enthusiasmus
            \item Wireless-Adhoc-Mesh Netzwerk
        \end{itemize}
        \item Offenes WLAN
        \item Nicht nur \glqq{}kostenloses Internet\grqq $\Rightarrow$ \glqq{}freies Netzwerken\grqq\\
        \begin{itemize}
            \item Unabhängiges Netz
            \item Internet Technologie beim Bürger
        \end{itemize}
        \item \texttt{(Irgendwann mal Teil vom Internet)}
    \end{itemize}
\end{frame}

\begin{frame}{Das braucht man}
    \begin{itemize}
        \item Einen günstigen, unterstützten Router (ab ca. 17€)
        \item Eine spezielle Firmware
        \item Die Zustimmung zum \glqq{}Pico-Peering Agreement\grqq
            \footnote{Regelwerk, über grundsätzliche Eigenschaften des Freifunks}:
            \begin{enumerate}
                \item Freier Transit
                \item Offene Kommunikation
                \item Keine Garantie (Haftungsausschluss)
                \item Nutzungsbestimmungen
                \item Lokale (individuelle) Zusätze
            \end{enumerate}
    \end{itemize}
\end{frame}

