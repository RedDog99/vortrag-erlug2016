\section{Netmon}

%Ein spezieller Dienst im Freifunk Franken Netz ist das
%Network-Monitoring, kurz Netmon. Netmon selbst soll nicht Teil des
%Vortrags sein, wohl aber die Verbindung zwischen Netmon und den
%Knoten. Dazu zählt zum Beispiel das Handling der Hostnames und das
%Einsammeln der Statusdaten.

\begin{frame}{Netmon}
    % hier mal kurz durch das Netmon durchklicken: Karte, Statistik, Routerliste, Routerstatus

    Knoten
    
    Netmon

    Nodewatcher

    Configurator

\end{frame}

\begin{frame}{Netmon}
    Was muss Netmon kennen, damit es den Router anzeigen kann?

    // Anforderungen an Netmon..

    \begin{itemize}
        \item IP-Adresse des Knotens
            ...
    \end{itemize}
\end{frame}


\begin{frame}{Nodewatcher}
    \begin{itemize}
        \item Erzeugt XML Datei
        \item Läuft alle 5 Minuten
        \item node.data über Webinterface downloadbar
    \end{itemize}
\end{frame}

\begin{frame}{Configurator}
    \begin{itemize}
        \item Knoten kennt Netmon's Link-Local Adresse
        \item Knoten meldet alle 5 Minuten seine MAC Adresse ans Netmon
        \item Netmon meldet dabei zurück, dass der Knoten noch nicht eingetragen wurde
        \item Benutzer ,,übernimmt'' Knoten im Netmon
        \item (Benutzer gibt dem Knoten einen Namen)
        \item Knoten meldet wieder seine MAC an Netmon
        \item Netmon meldet router\_id, update\_hash und api\_key zurück
        \item Knoten trägt nun seine Link-Local Addresse im Netmon ein
        \item Netmon pollt einmal alle 10 Minuten nach router-daten
        \item Knoten pollt alle 5 Minuten nach seinem Hostname
    \end{itemize}
\end{frame}
