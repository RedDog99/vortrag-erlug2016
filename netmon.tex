\section{Netmon}

%Ein spezieller Dienst im Freifunk Franken Netz ist das
%Network-Monitoring, kurz Netmon. Netmon selbst soll nicht Teil des
%Vortrags sein, wohl aber die Verbindung zwischen Netmon und den
%Knoten. Dazu zählt zum Beispiel das Handling der Hostnames und das
%Einsammeln der Statusdaten.

\begin{frame}{Netmon}
    \begin{itemize}
        \item Nodewatcher
        \begin{itemize}
            \item Generiert Status-Daten
            \item XML
            \item alle 5 Minuten
        \end{itemize}
        \item Configurator
        \begin{itemize}
            \item Verknüpft Netmon und Knoten
        \end{itemize}
        \item Crawler
        \begin{itemize}
            \item Sammelt Status-Daten
            \item Download über http Schnittstelle der Knoten
            \item Alles über Link-Local
        \end{itemize}
        \item Netmon
        \begin{itemize}
            \item Visualisiert Status-Daten
        \end{itemize}
    \end{itemize}
\end{frame}

\begin{frame}{Monitoring}
    \begin{itemize}
        \item Alfred
        \item todo Bild alfred masters auf jedem GW
    \end{itemize}
\end{frame}
