\section{Firmware}

%Nachdem der Aufbau der Firmware geklärt ist, soll ein kurzer Abriss
%über den Bauprozess der aktuellen Firmware gezeigt werden. Auch die
%letzten Entwicklungen der Franken eigenen Firmware werden gezeigt
%und erklärt. An dieser Stelle könnten Vorkenntnisse über OpenWrt
%Vorteilhaft sein, sind aber keine Voraussetzung für den Vortrag.

\begin{frame}[fragile]{Firmware}
    \only<1-3>{
        \begin{block}{Wie wird die Firmware gebaut?}
            \begin{itemize}
                \item<1-> git clone \sout{git://git.freifunk-ol.de/ffol/user/reddog/firmware.git}
                \item<1-> cd firmware
                \item<1-> ./buildscript
                \item<3-> ./buildscript selectbsp bsp/board\_wr1043nd.bsp
                \item<3-> ./buildscript selectcommunity community/franken.cfg
                \item<3-> ./buildscript
            \end{itemize}
        \end{block}
    }

    \only<2>{
        \texttt{\scriptsize
            Please select a Board-Support-Package using:\\
            ./buildscript selectbsp
        }
    }

    \only<4>{
        \texttt{\scriptsize
            Working with bsp/board\_wr1043nd.bsp and community/franken.cfg\\[2ex]
            This is the Build Environment Script of the Freifunk Community\\
            Oldenburg.\\
            Usage: ./buildscript command\\
            command:\\
            \hspace{1cm}selectcommunity [communityfile]\\
            \hspace{1cm}selectbsp [bsp file]\\
            \hspace{1cm}prepare\\
            \hspace{1cm}config\\
            \hspace{1cm}build\\
            \hspace{1cm}flash\\
            \hspace{1cm}download\\[2ex]
            If you need help to one of these options just type\\
            ./buildscript command help\\
        }
    }
\end{frame}

\begin{frame}[fragile]{Community File}
    \begin{block}{community/franken.cfg:}
        \scriptsize
        \begin{lstlisting}[gobble=12]
            BATMAN_CHANNEL=1
            ESSID_AP=franken.freifunk.net
            ESSID_MESH=batman.franken.freifunk.net
            BSSID_MESH=02:CA:FF:EE:BA:BE
            NETMON_IP=fe80::ff:feee:1
            VPN_PROJECT=fff
            NTPD_IP=fe80::ff:feee:1%br-mesh
            UPGRADE_PATH=http://[fe80::ff:feee:1%br-mesh]/dev/firmware/current/
        \end{lstlisting}
    \end{block}
\end{frame}

\begin{frame}[fragile]{Board-Support-Package}
    \begin{block}{bsp/board\_wr740n.bsp:}
        \scriptsize
        \begin{lstlisting}[gobble=12]
            machine=wr740n
            target=$builddir/$machine

            board_prepare() {
            }
            board_prebuild() {
            }
            board_postbuild() {
                cp $target/bin/ar71xx/openwrt-ar71[..]-squashfs-*.bin ./bin/
            }
            board_flash() {
                echo "nothing implemented"
            }
            board_clean() {
                /bin/rm -rf $target bin/*$machine*
            }
        \end{lstlisting}
    \end{block}
\end{frame}

\begin{frame}{Board-Support-Package}
    \begin{block}{bsp/wr740n/}
        \begin{itemize}
            \item .config
            \item root\_file\_system/etc/
            \begin{itemize}
                \item rc.local.board
                \item config/
                \begin{itemize}
                    \item board
                    \item network
                    \item system
                \end{itemize}
                \item crontabs/
                \begin{itemize}
                    \item root
                \end{itemize}
            \end{itemize}
        \end{itemize}
    \end{block}
\end{frame}

\begin{frame}{Firmware}

    das BSP file wird als dot Script eingeladen
    aus dem community file wird dynamisch ein SED skript erzeugt, um die Templates mit den richtigen Werten zu füllen

    Was passiert beim prepare
        Sourcen Downloaden
        dabei werden alle Sourcen in einen Source Folder geladen, sofern diese dort nicht bereichts vorhanden sind.
        - OpenWRT
        - Sämtliche Packages
        -- ggfs werden Patches angewandt
        
        Ein ggfs altes Target wird gelöscht
        OpenWRT wird ins Target exportiert (kopiert)
        Eine OpenWRT Feed-Config wird mit dem lokalen Source Verzeichnis als Quelle angelegt
        Die Feeds werden geladen
        Spezielle Auswahl an Paketen wird geladen
        Patches werden angewandt
        board\_prepare wird aufgerufen (wird. z.B. für Patches für eine bestimmte HW verwendet)

    Was passiert beim build
        prebuild
        make im OpenWRT
        postbuild
        

    Was kann man mit config machen?

\end{frame}

Wie funktioniert das innen drinne .. sourcen werden geladen etc ..

wann werden patches gemacht

wann wird die config kopiert

was ist ein bsp

wie bau ich ein bsp

was wird sich in 0.5.0 ändern? <- neue bsps
