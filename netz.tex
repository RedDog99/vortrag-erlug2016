\section{Netz}
\begin{frame}{}
    \begin{center}
        Das Netz
     \end{center}
\end{frame}

\begin{frame}{Knoten VPN}
    \begin{itemize}
        \item Verwendetes VPN: fastd
        \begin{itemize}
            \item n:m VPN
        \end{itemize}
        \item Endpunkt Vermittlung über KeyXchange:
        \begin{itemize}
            \item Zentrale Webseite \only<2-3>{{\color{red}Problem!}}
            \item Knoten meldet sich mit Standort
            \item Geographisch nächste Hood wird zugewiesen (voronoi)
            \item Client bekommt Liste aller Server der Hood
        \end{itemize}
        \item<3> Keine weitere Unterscheidung zu welcher Hood der Knoten gehört
        \item<3> {\color{red}Problem!} Funk Verbindung zwischen den Hoods
    \end{itemize}
\end{frame}

\begin{frame}{VPN Server}
    \begin{itemize}
        \item VPN Server: In jeder Hood gibt es mehrere davon
        \item Hoodzuweisung manuell im KeyXchange
        \item DHCP
        \begin{itemize}
            \item Aktuell ausschließlich IPv4
            \item Unterschiedliche Latenzen: Ungleiche Server Auslastung
            \begin{itemize}
                \item[$\rightarrow$] Batman-Adv GW Selection
                \item Anpassung nach Traffic Auslastung
            \end{itemize}
        \end{itemize}
        \item DNS Namesauflösung
        \item Policy base routing
        \item VPN (GRE) Tunnel zu anderen Gateways
        \begin{itemize}
            \item OLSR Routing
        \end{itemize}
    \end{itemize}
\end{frame}

\begin{frame}{Gateways}
    \begin{itemize}
        \item Verbindet Freifunk und Internet
        \item IPv4 NAT (oft übers Ausland) ins Internet
        \item Announced 0.0.0.0/0 via OLSR
        \begin{itemize}
            \item Dynamic Gateway Plugin
            \item VPN Server können diese Routen nutzen
        \end{itemize}
        \item Routing Metrik ohne Traffic/Bandbreite
        \item<2> {\color{red}Problem!} Ungleiche Traffic Verteilung
    \end{itemize}
\end{frame}
